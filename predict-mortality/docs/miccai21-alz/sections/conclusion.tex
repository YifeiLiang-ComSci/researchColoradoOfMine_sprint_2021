\iffalse
An important characteristic of the present study was the use of a semi-supervised classification method for the AD conversion prediction in MCI subjects. The semi-supervised method (LDS) was shown to outperform its counterpart supervised method (SVM) in the design of MRI biomarker.
\fi
\section{Conclusion}
% In this work we aim to detect the Alzheimer's Disease (AD) progress in its early stage from the longitudinal multi-modal healthcare dataset, 
We present a semi-supervised enrichment learning method that integrates the longitudinal multi-modal dataset and is clinically applicable for use in real-time, automatic AD diagnosis. The novel LSTM autoencoder compresses longitudinal records with missing data into a fixed-length vectorial representation. Armed with this enriched representation, one can fully utilize the genotypic and phenotypic data. Our experiments show that our model outperforms competing predictive models. When combined with the perturbation based feature identification method, our model also discovers the neuroimaging and genetic biomarkers associated with AD, adding further value to our approach.
% Our model aims to detect the Alzheimer's Disease (AD) progress in its early stage from the longitudinal multi-modal healthcare dataset, which fits to the clinical applicability and can be used in a real-time automatic AD diagnosis.
% We have conducted extensive experiments on the Alzheimer's Disease Neuroimaging Initiative (ADNI) dataset~\cite{risacher2010longitudinal} and 
% achieved promising performance on multiclass AD progression prediction task with the various number of labeled samples. In our experiments with the proposed model, we identify the disease-relevant biomarkers through the perturbation based feature identification method.