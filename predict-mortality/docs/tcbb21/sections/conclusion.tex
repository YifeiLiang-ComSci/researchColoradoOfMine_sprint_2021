\iffalse
We proposed models based on LSTMs that can learn good video representations.
First, given that the proposed machine learning method is purely data-driven, our model may vary if starting from different datasets. As more data become available, the whole procedure can easily be repeated to obtain more accurate models.
We does not assume any property about dataset.

In this paper, we considered the identification of COVID-19 cases from X-ray images and proposed a novel semi-supervised deep architecture that can distinguish between the three cases of Healthy, non-COVID pneumonia, COVID-19 infection based on the chest X-ray manifestation of these classes. The proposed methodology is comprised of two modules: 1) the Task-Based Feature Extraction Network (TFEN), and 2) the COVID-19 Identification Network (CIN). 
\fi
%%%%%%%%%%%%%%%%%%%%%%%%%%%%%%%%%%%%%%%%%%%%%%%%%%%%%%%%%%%%%%%%%%%%%%%%%%%%%%%%%%%
\section{Conclusion}
We propose a semi-supervised enrichment method based on a novel LSTM autoencoder that is clinicaly applicable and can make real-time automatic mortality predictions. The enriched representation of MTS data is in a fixed-length vector format and can be readily integrated with the static data. In our experiments and case studies, the proposed model shows state-of-the-art performance in predicting mortality as well as increased flexability in handeling labeled and unlabled data. Additionly, when combined with the perturbation based feature identification method, our model identifies the risk factors of mortality that are consistant with the findings of previous medical studies and predictive models. 
% Our model is purely data-driven, which means the prediction performance can be further improved with additional data. 
Since there is no assumption or limitation in the dataset property, this research proposes the general framework to fully utilize the MTS dataset, and other models stemming from our enrichment approach are able to perform different prediction tasks.
\clearpage