\iffalse
Mild cognitive impairment (MCI) is a transitional stage between age-related cognitive decline and Alzheimer’s disease (AD). For the effective treatment of AD, it would be important to identify MCI patients at high risk for conversion to AD. 
The novel characteristics of the methods for learning the biomarkers are as follows: 1) We used a semi-supervised learning method (low density separation) for the construction of MRI biomarker as opposed to more typical supervised methods; 2)
4) We constructed the aggregate biomarker by first learning a separate MRI biomarker and then combining it with age and cognitive measures about the MCI subjects at the baseline by applying a random forest classifier.
Alzheimer’s disease (AD), a common form of dementia, occurs most frequently in aged population. More than 30 million people worldwide suffer from AD and, due to the increasing life expectancy, this number is expected to triple by 2050 (The projected effect of risk factor reduction on Alzheimer's disease prevalence, Barnes DE).
Because of the dramatic increase in the prevalence of AD, the identification of effective biomarkers for the early diagnosis and treatment of AD in individuals at high risk to develop the disease is crucial.
Mild cognitive impairment (MCI) is a transitional stage between age-related cognitive decline and AD, and the earliest clinically detectable stage of progression towards dementia or AD (Neuropathologic alterations in mild cognitive impairment: a review, Markesbery WR). 
AD pathology has been therefore hypothesized to be detectable using neuroimaging techniques.(Neuropathologic alterations in mild cognitive impairment: a review, Markesbery WR)
Among different neuroimaging modalities, MRI has attracted a significant interest in AD related studies because of its completely non-invasive nature, high availability, high spatial resolution and good contrast between different soft tissues.
Clearly, predicting this disease in the early stages and preventing it from progressing is of great importance.
The diagnosis of Alzheimer’s disease (AD) requires a variety of medical tests, which leads to huge amounts of multivariate heterogeneous data. It can be difficult and exhausting to manually compare, visualize, and analyze this data due to the heterogeneous nature of medical tests; therefore, an efficient approach for accurate prediction of the condition of the brain through the classification of magnetic resonance imaging (MRI) images is greatly beneficial and yet very challenging. In this paper, a novel approach is proposed for the diagnosis of very early stages of AD through an efficient classification of brain MRI images
Next, using a small set of labeled training data,
The reason that early diagnosis of AD is of great importance is that the clinical therapies given to patients are much more effective in slowing down disease progression and helping preserve some cognitive functions of the brain if the patients are in the early stages of their disease.
In this paper, we propose a novel method for a high-level latent and shared feature representation from neuroimaging modalities via deep learning. 
Furthermore, fusing the complementary information from multiple modalities helps enhance the diagnostic accuracy[Identification of conversion from mild cognitive impairment to Alzheimer's disease using multivariate predictors.; Multivariate examination of brain abnormality using both structural and functional MRI.]
To this end, many researchers have devoted their efforts to find biomarkers and develop a computer-aided system, with which we can effectively predict or diagnose the diseases.
In a semi-supervised method, feature vectors from unlabeled data are also used in the learning process in addition to the labels and feature vectors from the labeled ones.
\fi

\section{Introduction}
% Why our model is good? : (1) Semi-supervised learning -> Our model can learn from labeled/unlabeled samples BOTH. (2) Learn temporal trends from the time series with missing records (MRI imagings are captured inconsistently (the number of records/the time points of records are different) from the participants). (3) Combine the multi-modal (genotypic modality (SNPs, static) + phenotypic (neuroimagings, dynamic) modality) data with two Autoencoders: one for static, another for time series.
% Find the related works with the lack of above advantages of our model.