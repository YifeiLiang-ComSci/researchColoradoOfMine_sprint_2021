% This version of CVPR template is provided by Ming-Ming Cheng.
% Please leave an issue if you found a bug:
% https://github.com/MCG-NKU/CVPR_Template.

\documentclass[review]{cvpr}
%\documentclass[final]{cvpr}

\usepackage{times}
\usepackage{epsfig}
\usepackage{graphicx}
\usepackage{amsmath}
\usepackage{amssymb}

% Include other packages here, before hyperref.

% If you comment hyperref and then uncomment it, you should delete
% egpaper.aux before re-running latex.  (Or just hit 'q' on the first latex
% run, let it finish, and you should be clear).
\usepackage[pagebackref=true,breaklinks=true,colorlinks,bookmarks=false]{hyperref}


\def\cvprPaperID{****} % *** Enter the CVPR Paper ID here
\def\confYear{CVPR 2021}
%\setcounter{page}{4321} % For final version only


\begin{document}

%%%%%%%%% TITLE
\title{Learning Deep Neuroimaging Representation by Longitudinal Augmentations}

\author{First Author\\
Institution1\\
Institution1 address\\
{\tt\small firstauthor@i1.org}
% For a paper whose authors are all at the same institution,
% omit the following lines up until the closing ``}''.
% Additional authors and addresses can be added with ``\and'',
% just like the second author.
% To save space, use either the email address or home page, not both
\and
Second Author\\
Institution2\\
First line of institution2 address\\
{\tt\small secondauthor@i2.org}
}

\maketitle


%%%%%%%%% ABSTRACT
\begin{abstract}
   Alzheimer's Disease (AD) is a chronic neurodegenerative disease that severely causes problems on patient's thinking, memory, and behavior. As early diagnosis is important to prevent AD progression, many biomarker analysis models have been presented to predict clinical outcomes. However, these models often fail to integrate heterogeneous genotypic and phenotype biomarkers to improve diagnosis prediction and/or are not able to deal with the incomplete longitudinal biomarkers with missing entries/records. Thus we propose the semi-supervised augmentation learning method to learn a abstract vectorial representation of multimodal longitudinal biomarkers. We use a composite model of  autoencoders based on Recurrent Neural Network. Our experiments show that our augmentation model improves the prediction performance on AD progression.
\end{abstract}

%%%%%%%%% BODY TEXT
%% --- Useful instructions from template:
% Papers, excluding the references section,
% must be no longer than eight pages in length. The references section
% will not be included in the page count, and there is no limit on the
% length of the references section. For example, a paper of eight pages
% with two pages of references would have a total length of 10 pages.

% Please number all of your sections and displayed equations.~\cite{Authors14}

% \begin{figure}[t]
%    \begin{center}
%    \fbox{\rule{0pt}{2in} \rule{0.9\linewidth}{0pt}}
%       %\includegraphics[width=0.8\linewidth]{egfigure.eps}
%    \end{center}
%       \caption{Example of caption.  It is set in Roman so that mathematics
%       (always set in Roman: $B \sin A = A \sin B$) may be included without an
%       ugly clash.}
%    \label{fig:long}
%    \label{fig:onecol}
% \end{figure}

\iffalse
Mild cognitive impairment (MCI) is a transitional stage between age-related cognitive decline and Alzheimer’s disease (AD). For the effective treatment of AD, it would be important to identify MCI patients at high risk for conversion to AD. 
The novel characteristics of the methods for learning the biomarkers are as follows: 1) We used a semi-supervised learning method (low density separation) for the construction of MRI biomarker as opposed to more typical supervised methods; 2)
4) We constructed the aggregate biomarker by first learning a separate MRI biomarker and then combining it with age and cognitive measures about the MCI subjects at the baseline by applying a random forest classifier.
Alzheimer’s disease (AD), a common form of dementia, occurs most frequently in aged population. More than 30 million people worldwide suffer from AD and, due to the increasing life expectancy, this number is expected to triple by 2050 (The projected effect of risk factor reduction on Alzheimer's disease prevalence, Barnes DE).
Because of the dramatic increase in the prevalence of AD, the identification of effective biomarkers for the early diagnosis and treatment of AD in individuals at high risk to develop the disease is crucial.
Mild cognitive impairment (MCI) is a transitional stage between age-related cognitive decline and AD, and the earliest clinically detectable stage of progression towards dementia or AD (Neuropathologic alterations in mild cognitive impairment: a review, Markesbery WR). 
AD pathology has been therefore hypothesized to be detectable using neuroimaging techniques.(Neuropathologic alterations in mild cognitive impairment: a review, Markesbery WR)
Among different neuroimaging modalities, MRI has attracted a significant interest in AD related studies because of its completely non-invasive nature, high availability, high spatial resolution and good contrast between different soft tissues.
Clearly, predicting this disease in the early stages and preventing it from progressing is of great importance.
The diagnosis of Alzheimer’s disease (AD) requires a variety of medical tests, which leads to huge amounts of multivariate heterogeneous data. It can be difficult and exhausting to manually compare, visualize, and analyze this data due to the heterogeneous nature of medical tests; therefore, an efficient approach for accurate prediction of the condition of the brain through the classification of magnetic resonance imaging (MRI) images is greatly beneficial and yet very challenging. In this paper, a novel approach is proposed for the diagnosis of very early stages of AD through an efficient classification of brain MRI images
Next, using a small set of labeled training data,
The reason that early diagnosis of AD is of great importance is that the clinical therapies given to patients are much more effective in slowing down disease progression and helping preserve some cognitive functions of the brain if the patients are in the early stages of their disease.
In this paper, we propose a novel method for a high-level latent and shared feature representation from neuroimaging modalities via deep learning. 
Furthermore, fusing the complementary information from multiple modalities helps enhance the diagnostic accuracy[Identification of conversion from mild cognitive impairment to Alzheimer's disease using multivariate predictors.; Multivariate examination of brain abnormality using both structural and functional MRI.]
To this end, many researchers have devoted their efforts to find biomarkers and develop a computer-aided system, with which we can effectively predict or diagnose the diseases.
In a semi-supervised method, feature vectors from unlabeled data are also used in the learning process in addition to the labels and feature vectors from the labeled ones.
\fi

\section{Introduction}
% Why our model is good? : (1) Semi-supervised learning -> Our model can learn from labeled/unlabeled samples BOTH. (2) Learn temporal trends from the time series with missing records (MRI imagings are captured inconsistently (the number of records/the time points of records are different) from the participants). (3) Combine the multi-modal (genotypic modality (SNPs, static) + phenotypic (neuroimagings, dynamic) modality) data with two Autoencoders: one for static, another for time series.
% Find the related works with the lack of above advantages of our model.
\section{Experiment}
Experiment here.
\iffalse
We proposed models based on LSTMs that can learn good video representations.
First, given that the proposed machine learning method is purely data-driven, our model may vary if starting from different datasets. As more data become available, the whole procedure can easily be repeated to obtain more accurate models.
We does not assume any property about dataset.

In this paper, we considered the identification of COVID-19 cases from X-ray images and proposed a novel semi-supervised deep architecture that can distinguish between the three cases of Healthy, non-COVID pneumonia, COVID-19 infection based on the chest X-ray manifestation of these classes. The proposed methodology is comprised of two modules: 1) the Task-Based Feature Extraction Network (TFEN), and 2) the COVID-19 Identification Network (CIN). 
\fi
%%%%%%%%%%%%%%%%%%%%%%%%%%%%%%%%%%%%%%%%%%%%%%%%%%%%%%%%%%%%%%%%%%%%%%%%%%%%%%%%%%%
\section{Conclusion}
We propose a semi-supervised enrichment method based on a novel LSTM autoencoder that is clinicaly applicable and can make real-time automatic mortality predictions. The enriched representation of MTS data is in a fixed-length vector format and can be readily integrated with the static data. In our experiments and case studies, the proposed model shows state-of-the-art performance in predicting mortality as well as increased flexability in handeling labeled and unlabled data. Additionly, when combined with the perturbation based feature identification method, our model identifies the risk factors of mortality that are consistant with the findings of previous medical studies and predictive models. 
% Our model is purely data-driven, which means the prediction performance can be further improved with additional data. 
Since there is no assumption or limitation in the dataset property, this research proposes the general framework to fully utilize the MTS dataset, and other models stemming from our enrichment approach are able to perform different prediction tasks.
\clearpage

{\small
\bibliographystyle{ieee_fullname}
\bibliography{bibliography}
}

\end{document}
