\iffalse
We proposed models based on LSTMs that can learn good video representations.
First, given that the proposed machine learning method is purely data-driven, our model may vary if starting from different datasets. As more data become available, the whole procedure can easily be repeated to obtain more accurate models.
We does not assume any property about dataset.

In this paper, we considered the identification of COVID-19 cases from X-ray images and proposed a novel semi-supervised deep architecture that can distinguish between the three cases of Healthy, non-COVID pneumonia, COVID-19 infection based on the chest X-ray manifestation of these classes. The proposed methodology is comprised of two modules: 1) the Task-Based Feature Extraction Network (TFEN), and 2) the COVID-19 Identification Network (CIN). 
\fi
%%%%%%%%%%%%%%%%%%%%%%%%%%%%%%%%%%%%%%%%%%%%%%%%%%%%%%%%%%%%%%%%%%%%%%%%%%%%%%%%%%%
\section{Conclusion}
We propose semi-supervised enrichment methods based on the novel LSTM autoencoder which fits to the clinical applicability and can be used in a real-time automatic mortality prediction. The enriched representation in the vectorial format summarizes the uneven and incomplete MTS data. Armed with the enriched representation, one can fully utilize the prediction capability of the MTS dataset. In our experiments, the proposed model shows the state-of-the-art prediction performance on mortality prediction task. In addition, combined with the perturbation based feature identification method, our model discovers the most predictive biomarkers, further adding value to our approach in the interpretability aspect. Our model is purely data-driven, which means the prediction performance can be further improved with additional data. Since there is no assumption or limitation in the dataset property, the other models stemming from our enrichment approach are able to perform the different prediction tasks with the various MTS dataset.
\clearpage