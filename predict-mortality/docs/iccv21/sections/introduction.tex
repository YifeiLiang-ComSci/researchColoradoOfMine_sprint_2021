\section{Introduction}
\iffalse 

1. Which problem we are going to solve?
2. Why solving this problem is important?
3. Which studies have beed tried to solve this problem and what are the disadvantages/difficulties of them?
    3-1. Which properties of longitudinal image datasets should be considered to solve the problem? (e.g. time series, inconsistent number of scans for different participants, some are labeled others are not labeled, ..)
4. So which model we are going to propose to solve the listed problems above?
    4-1. Learn enriched representation of longitudinal images data in a fixed-length vector format which can be readily integrated with the static data.
    4-2. Convolutional layer + LSTM to enrich the images.
    4-3. The proposed model is Semi-supervised learning model.~\cite{tuncel2018autoregressive}
\fi 

Sudden increases in COVID-19 cases are incredibly stressful on healthcare systems and quickly deplete the all ready limited resources of hospitals~\cite{centersstrategies}. Clinicians are forced to make difficult descisons about the distribution of scarce treatments without reliable predictions about patients' conditions. To aid in this problem, machine learning models have been proposed to inform logistical planning such as in the study by~\cite{yan2020interpretable}. This model \cite{yan2020interpretable} identifies the most predictive biomarkers for a patient's mortality using a XGBoost classifier~\cite{chen2016xgboost} trained on a publicly available MTS dataset collected from COVID-19-positive patients admitted to Tongji Hospital in China. Due to the limitations of random XGBoost classifiers in synthesizing longitudinal data, only the final record was used to train and test the model. While the model effectively determines the most predictive biomarkers, it hardly captures the temporal trends, which are critical in determining the progression of the disease. It also fails to predict the mortality when those principle biomarkers are not measured. An intrinsic problem of the COVID-19 pandemic is that samples from patients are rarely collected in a uniform way. X-rays of patients are not preformed for a controlled study, resulting in a different number for every patient as well as irregular time intervals between samples. MTS analysis models have been proposed to learn relationships across variables and time stamps, however, conventional MTS analysis often requires sequences to be complete and of consistent length~\cite{lu2018predicting,wang2017longitudinal,wang2016prediction,wang2012high} which is problematic for the reasons stated above. Furthermore, datasets collected out of carefully controlled environments tend to have missing data which is particularly problematic when that includes patient outcome.

% The impact of the novel COVID-19 virus has heavily affected the global environment. Despite its lower mortality rate, the disease has higher death totals then similar but more lethal diseases such as SARS and MERS [1]. In addition to lives lost, the strain on institutional health care systems has been massive, where the disease has induced critical supply storages in personal and protective equipment [2]. The cumulation of these two aspects creates a vital issue to be addressed, and the ability to predict mortality based on biomarkers could be essential towards forward progress. Many researchers have been pooling efforts into creating an accurate and effective COVID-19 mortality predictor, and previous models have had success [blood covid analysis] with a combination of semi-supervised LSTM and auto-encoder neural networks. The findings take into account temporal trends with fixed features to accurately predict the mortality rate based upon blood samples from the Tongji Hospital in China. Even though this approach was successful, its restriction to being only on blood samples constrains the scope of its application, and thus our goal is to apply a similar machine learning approach to COVID-19 X-Ray dataset. This dataset is a summation of chest X-rays consisting of 257 frontal chest X-rays and 20 CT scans in NIFTI format. It is not directly sourced from hospital researchers, as this is unfortunately nearly impossible, but it was obtained web scrapping of hospital publications which were then complied together. Underlying patterns have been shown to appear within the covid-only chest X-ray dataset, suchthat covid patients showed abnormalities in chest CT images with most having bilateral involvement [3]. Thus, by the time series of patient x-rays could provide valuable information in the mortality predictor. Similarly, to other non-controlled time series data collection, a plethora of the series are incomplete, and the data as a whole has a significant number of holes. Thus, a semi-supervised approach must be taken to be able to derive any significant conclusions from the given dataset. In addition to the time series, there are static features that must be accounted for as well. These problems were similarly addressed in the models mentioned beforehand, as they combine an autoencoder for the static features and a LSTM to account for the time series data. Our model will use this basis but with a clear distinction. The X-ray data is 2-D dimensional and must be processed, and thus we will use a Convolutional neural network to create an enriched vector as input for theLSTM. Also, a deconvolutional neural network must be added to calculate the respective loss function. The summation of both the Convolutional network and the LSTM creates highly enriched vectors for our generator function to produce effectiveresults.[1] doi: https://doi.org/10.1136/bmj.m641[2] DOI:https://doi.org/10.1016/S0140-6736(20)30183-5[3] DOI: 10.1056/NEJMp2006141
