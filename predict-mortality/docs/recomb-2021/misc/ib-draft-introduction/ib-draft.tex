\def\year{2021}\relax
%File: formatting-instructions-latex-2021.tex
%release 2021.1
\documentclass[letterpaper]{article} % DO NOT CHANGE THIS
\usepackage{aaai21}  % DO NOT CHANGE THIS
\usepackage{times}  % DO NOT CHANGE THIS
\usepackage{helvet} % DO NOT CHANGE THIS
\usepackage{courier}  % DO NOT CHANGE THIS
\usepackage[hyphens]{url}  % DO NOT CHANGE THIS
\usepackage{graphicx} % DO NOT CHANGE THIS
\urlstyle{rm} % DO NOT CHANGE THIS
\def\UrlFont{\rm}  % DO NOT CHANGE THIS
\usepackage{natbib}  % DO NOT CHANGE THIS AND DO NOT ADD ANY OPTIONS TO IT
\usepackage{caption} % DO NOT CHANGE THIS AND DO NOT ADD ANY OPTIONS TO IT
\frenchspacing  % DO NOT CHANGE THIS
\setlength{\pdfpagewidth}{8.5in}  % DO NOT CHANGE THIS
\setlength{\pdfpageheight}{11in}  % DO NOT CHANGE THIS
%\nocopyright
%PDF Info Is REQUIRED.
% For /Author, add all authors within the parentheses, separated by commas. No accents or commands.
% For /Title, add Title in Mixed Case. No accents or commands. Retain the parentheses.
\pdfinfo{
/Title (Unsupervised Embedding via Locality Preserving Autoencoder to Predict Mortality in Patients with COVID-19)
/Author (Hoon Seo, Hua Wang)
} %Leave this
% /Title ()
% Put your actual complete title (no codes, scripts, shortcuts, or LaTeX commands) within the parentheses in mixed case
% Leave the space between \Title and the beginning parenthesis alone
% /Author ()
% Put your actual complete list of authors (no codes, scripts, shortcuts, or LaTeX commands) within the parentheses in mixed case.
% Each author should be only by a comma. If the name contains accents, remove them. If there are any LaTeX commands,
% remove them.

%% Load Packages
\usepackage{amsmath}

% DISALLOWED PACKAGES
% \usepackage{authblk} -- This package is specifically forbidden
% \usepackage{balance} -- This package is specifically forbidden
% \usepackage{color (if used in text)
% \usepackage{CJK} -- This package is specifically forbidden
% \usepackage{float} -- This package is specifically forbidden
% \usepackage{flushend} -- This package is specifically forbidden
% \usepackage{fontenc} -- This package is specifically forbidden
% \usepackage{fullpage} -- This package is specifically forbidden
% \usepackage{geometry} -- This package is specifically forbidden
% \usepackage{grffile} -- This package is specifically forbidden
% \usepackage{hyperref} -- This package is specifically forbidden
% \usepackage{navigator} -- This package is specifically forbidden
% (or any other package that embeds links such as navigator or hyperref)
% \indentfirst} -- This package is specifically forbidden
% \layout} -- This package is specifically forbidden
% \multicol} -- This package is specifically forbidden
% \nameref} -- This package is specifically forbidden
% \usepackage{savetrees} -- This package is specifically forbidden
% \usepackage{setspace} -- This package is specifically forbidden
% \usepackage{stfloats} -- This package is specifically forbidden
% \usepackage{tabu} -- This package is specifically forbidden
% \usepackage{titlesec} -- This package is specifically forbidden
% \usepackage{tocbibind} -- This package is specifically forbidden
% \usepackage{ulem} -- This package is specifically forbidden
% \usepackage{wrapfig} -- This package is specifically forbidden
% DISALLOWED COMMANDS
% \nocopyright -- Your paper will not be published if you use this command
% \addtolength -- This command may not be used
% \balance -- This command may not be used
% \baselinestretch -- Your paper will not be published if you use this command
% \clearpage -- No page breaks of any kind may be used for the final version of your paper
% \columnsep -- This command may not be used
% \newpage -- No page breaks of any kind may be used for the final version of your paper
% \pagebreak -- No page breaks of any kind may be used for the final version of your paperr
% \pagestyle -- This command may not be used
% \tiny -- This is not an acceptable font size.
% \vspace{- -- No negative value may be used in proximity of a caption, figure, table, section, subsection, subsubsection, or reference
% \vskip{- -- No negative value may be used to alter spacing above or below a caption, figure, table, section, subsection, subsubsection, or reference

\setcounter{secnumdepth}{0} %May be changed to 1 or 2 if section numbers are desired.

% The file aaai21.sty is the style file for AAAI Press
% proceedings, working notes, and technical reports.
%

% Title

% Your title must be in mixed case, not sentence case.
% That means all verbs (including short verbs like be, is, using,and go),
% nouns, adverbs, adjectives should be capitalized, including both words in hyphenated terms, while
% articles, conjunctions, and prepositions are lower case unless they
% directly follow a colon or long dash

\title{Unsupervised Embedding via Locality Preserving Autoencoder for Improved Mortality Prediction in Patients with COVID-19}
\author{
    %Authors
    % All authors must be in the same font size and format.
    Anonymous Authors
}

\begin{document}

\maketitle

\begin{abstract} 
An influx of COVID-19 infections puts pressure on healthcare systems and
    disrupts their general care routine, leading to an increase in excess
    deaths. Efficient allocation of finite resources, is a persistent problem
    most apparent during a sudden influx of patients. Earlier studies have
    suggested the use of biomarkers, particularly those measured in blood
    samples, to build mortality-prediction models to aid in logistical
    decision-making. These same studies focused on isolating predictive
    biomarkers rather than building a deterministic model. This paper uses
    longitudinal data collected from blood samples of 375 patients infected
    with COVID-19 in Wuhan, China, to produce a general predictive
    mortality-prediction models. We propose an embedding framework that
    summarizes sparse time series measurements into a single fixed-length
    vector by combining locality preserving projections and deep learning.
    Models equipped with this new framework promise significantly higher
    prediction accuracy in addition to training on less data.
\end{abstract}

\section{Introduction}
Sudden increases in COVID-19 cases, such as during seasonal waves, quickly
deplete the limited resources of healthcare systems~\cite{kontis2020magnitude}, forcing clinicians to set
criteria for distribution of scarce treatments, such as N95
respirators~\cite{centers2020strategies}. In an earlier study, Yan et al. advocated for
a machine learning mortality-prediction model to inform logistical
planning~\cite{yan2020interpretable}.  The model proposed in the study takes in data from
patients' blood samples to identify the most predictive biomarkers using
a random forest algorithm. This data was obtained from a publicly available
dataset whose samples were collected using case report forms (CRFs) and
contained longitudinal records on 358 COVID-19-positive patients admitted to
Tongji Hospital in China. Most patient samples also contained a `patient
outcome' feature, which the model could use as a target variable. Acknowledging
the limitations of random forest models in taking in misaligned temporal data,
only the final record from each patient was used to train and test the model.
While this technique accomplished the task of deter- mining the most predictive
biomarkers, it failed to capture change in time, a feature central to the
progression of COVID-19. In this study, we attempt a similar classification
task on this same dataset, but with the goal of producing a general predictive
model that incorporates temporal features into its predictions.

Conventional classification methods in time series analysis require vector
sequences of fixed length, which is rarely the case when data is collected
outside of a carefully controlled environment. Our samples were collected
during an outbreak of a pandemic, a time when hospitals function out of routine
blood draws are not performed with for a controlled study. As a result, the
number of blood draws different for every patient and  time intervals between
samples are also often irregular. The data for this specific dataset was also
conducted using CRFs, or surveys filled out by individuals rather than being
transcribed directly from a test result. The resultant inputs are uneven
sequences of vectors with missing information that introduce several new
difficulties, particularly when missing the missing information includes
patient outcome, the target variable for any predictive model.

One might suggest imputation as a possible solution, at least to tackle missing
the missing data problem. Modern imputation methods rely on a framework of
generative adversarial networks (GANs)~\cite{yoon2018gain} to successfully impute
sizeable portions of high-dimensional data. Although they are efficient, even
state-of-the-art techniques assume records to be missing-at-random, an
assumption that risks introducing unintended bias into a model. If we suppose
that the data is perfectly imputed and that we are able to coerce
a classification model to recognize time as a feature, there is still
variability within individual samples. Patient blood is sampled only when
the clinician deems it necessary, so the number of records in a sequence, as
well as the time interval between samples, is most likely different for every patient.
Many time series models require the number of the records to be uniform across
sequences while techniques that are reliant on time underlying time functions,
such as autoencoders, fail completely when the intervals between records are
non-uniform~\cite{yu2013embedding}. All of these difficulties are typically tackled by
removing samples, a practice that leads to a smaller dataset and a less
accurate model.

\newpage
\bibliographystyle{aaai21}
\bibliography{master}
\end{document}
