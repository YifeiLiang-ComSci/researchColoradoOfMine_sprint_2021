\iffalse
We proposed models based on LSTMs that can learn good video representations.
First, given that the proposed machine learning method is purely data-driven, our model may vary if starting from different datasets. As more data become available, the whole procedure can easily be repeated to obtain more accurate models.
We does not assume any property about dataset.

In this paper, we considered the identification of COVID-19 cases from X-ray images and proposed a novel semi-supervised deep architecture that can distinguish between the three cases of Healthy, non-COVID pneumonia, COVID-19 infection based on the chest X-ray manifestation of these classes. The proposed methodology is comprised of two modules: 1) the Task-Based Feature Extraction Network (TFEN), and 2) the COVID-19 Identification Network (CIN). 
\fi
%%%%%%%%%%%%%%%%%%%%%%%%%%%%%%%%%%%%%%%%%%%%%%%%%%%%%%%%%%%%%%%%%%%%%%%%%%%%%%%%%%%
\section{Conclusion}
We propose semi-supervised enrichment methods based on an LSTM autoencoder. The enriched representation in a fixed length vectorial format summarizes time series data with uneven time intervals and possible missing entries, allowing the decoder to explicitly utilize the uneven time intervals to reconstruct the original data. Our model is able to learn from both labeled or unlabeled samples both jointly, producing state-of-the-art prediction performance on mortality prediction. Armed with the trained model, our model can readily predict the mortality of a new patient free from the cold start problem. Our model is purely data-driven, which means the prediction performance can be further improved with additional data. Since there is no assumption or limitation in predictor structure or target labels, the other models stemming from our approach will be flexible to solve the other problems, such as multi-label classification or regression problem.
\clearpage